\section{Conclusion and Future Work}
This study deconstructed Flow Forecaster, revealing an industrial-strength integration of Monte Carlo simulation and machine learning for software project forecasting. Through repository mining and empirical execution of the bundled regression suites, we validated that the platform delivers statistically coherent percentile forecasts, ensemble-based accuracy tracking, and interpretability features such as data-quality diagnostics and targeted recommendations. The Monte Carlo engine consistently replicated probabilistic capacity tables and deadline analyses, while the ML subsystem executed cross-validated training across nine regressors with walk-forward evaluation, confirming the feasibility of a unified experimentation surface for practitioners and researchers alike.

Nevertheless, the investigation remains bounded by the synthetic datasets packaged with the repository and the limited runtime context provided by test scripts. We could not observe real industrial traces, longitudinal deployments, or user feedback, which constrains claims regarding external validity and operational robustness. Future work should pursue (i) controlled studies using anonymized project datasets from diverse domains to benchmark the platform against competing tools, (ii) ablation analyses that quantify the incremental benefit of each diagnostic module (e.g., dependency adjustments, risk summaries), and (iii) integration of explainable AI techniques to surface feature attributions for throughput predictions. Collecting richer telemetry---such as actual forecast vs.\ outcome logs, user interaction patterns, and computational cost measurements---would further strengthen the empirical foundation and guide performance optimization in large-scale deployments.
