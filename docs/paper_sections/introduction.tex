\section{Introduction}
Software-intensive organizations increasingly rely on quantitative forecasting to balance delivery commitments, staffing, and risk exposure. Yet contemporary practice still oscillates between spreadsheet-driven Monte Carlo simulations and opaque machine learning (ML) models, often maintained as separated toolchains with little empirical integration. Flow Forecaster---the subject of this study---emerges as an open, end-to-end platform that unifies both paradigms inside a single Flask-based web system. By coupling classical Monte Carlo sampling with ML ensembles that operate on historical throughput data, the tool promises faster what-if analysis, richer risk diagnostics, and an auditable experimentation surface for software engineering teams.

Despite the platform's rapid adoption in practitioner communities, the peer-reviewed literature lacks a rigorous characterization of its architecture, evaluation procedures, and empirical evidence. Forecasting research in software project management has long advocated for replicable pipelines that disclose data preparation, hyperparameter strategies, and validation metrics. Without such analyses, it remains unclear whether Flow Forecaster's claims about accuracy, explainability, and operational readiness hold under scrutiny.

This paper addresses that gap by reverse-engineering the publicly available code base, reconstructing its experimental design, and systematically cataloging the Monte Carlo and ML capabilities embedded in the implementation. We provide three contributions. First, we delineate the system's architecture, highlighting how Monte Carlo and ML pipelines share feature engineering, dependency modeling, and visualization components. Second, we document the experimentation workflow, including cross-validation, backtesting, and regression tests that assess statistical consistency. Third, we synthesize the quantitative and qualitative findings surfaced by the internal analytics modules, positioning the platform as a research-grade environment for comparative forecasting studies. Collectively, these insights advance the empirical understanding of hybrid forecasting tooling in software engineering.
